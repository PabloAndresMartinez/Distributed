%%%%%%%%%%%%%%%%%%%%%%%%
%% Sample use of the infthesis class to prepare a thesis. This can be used as 
%% a template to produce your own thesis.
%%
%% The title, abstract and so on are taken from Martin Reddy's csthesis class
%% documentation.
%%
%% MEF, October 2002
%%%%%%%%%%%%%%%%%%%%%%%%

%%%%
%% Load the class. Put any options that you want here (see the documentation
%% for the list of options). The following are samples for each type of
%% thesis:
%%
%% Note: you can also specify any of the following options:
%%  logo: put a University of Edinburgh logo onto the title page
%%  frontabs: put the abstract onto the title page
%%  deptreport: produce a title page that fits into a Computer Science
%%      departmental cover [not sure if this actually works]
%%  singlespacing, fullspacing, doublespacing: choose line spacing
%%  oneside, twoside: specify a one-sided or two-sided thesis
%%  10pt, 11pt, 12pt: choose a font size
%%  centrechapter, leftchapter, rightchapter: alignment of chapter headings
%%  sansheadings, normalheadings: headings and captions in sans-serif
%%      (default) or in the same font as the rest of the thesis
%%  [no]listsintoc: put list of figures/tables in table of contents (default:
%%      not)
%%  romanprepages, plainprepages: number the preliminary pages with Roman
%%      numerals (default) or consecutively with the rest of the thesis
%%  parskip: don't indent paragraphs, put a blank line between instead
%%  abbrevs: define a list of useful abbreviations (see documentation)
%%  draft: produce a single-spaced, double-sided thesis with narrow margins
%%
%% For a PhD thesis -- you must also specify a research institute:
\documentclass[mscres, lfcs, logo]{infthesis}

%% For an MPhil thesis -- also needs an institute
% \documentclass[mphil,ianc]{infthesis}

%% MSc by Research, which also needs an institute
% \documentclass[mscres,irr]{infthesis}

%% Taught MSc -- specify a particular degree instead. If none is specified,
%% "MSc in Informatics" is used.
% \documentclass[msc,cogsci]{infthesis}
% \documentclass[msc]{infthesis}  % for the MSc in Informatics

%% Master of Informatics (5 year degree)
% \documentclass[minf]{infthesis}

%% Undergraduate project -- specify the degree course and project type
%% separately
% \documentclass[bsc]{infthesis}
% \course{Artificial Intelligence and Psychology}
% \project{Fourth Year Project Report}

%% Put any \usepackage commands you want to use right here; the following is 
%% an example:
\usepackage{natbib}
\usepackage{amsfonts}
\usepackage{amsmath}
\usepackage{amssymb}
\usepackage{amsthm}
\usepackage{listings}
\usepackage{xcolor}
\usepackage{colortbl}
\usepackage{verbatim}
\usepackage{geometry}
\usepackage{marginnote}
\usepackage{ragged2e}
\usepackage{pgfplotstable}
\usepackage{pgfplots}
\usepackage{pifont}
\usepackage{multirow}
\usepackage{url}

\newcounter{nthm}[chapter] % defines algorithm counter for chapter-level
\renewcommand{\thenthm}{\thechapter .\arabic{nthm}} %defines appearance of the algorithm counter
\newtheorem{theorem}[nthm]{Theorem}
\newtheorem{corollary}[nthm]{Corollary}
\newtheorem{remark}[nthm]{Remark}

\usepackage{tikz}
\usetikzlibrary{positioning}
\usetikzlibrary{patterns}

\captionsetup{font=small,labelfont=bf}
\setlength{\abovecaptionskip}{0pt}

\newcounter{nalg}[chapter] % defines algorithm counter for chapter-level
\renewcommand{\thenalg}{\thechapter .\arabic{nalg}} %defines appearance of the algorithm counter
\DeclareCaptionLabelFormat{algocaption}{Algorithm \thenalg} % defines a new caption label as Algorithm x.y

\newcommand{\cmark}{\ding{51}}%

\lstnewenvironment{algorithm}[1][] %defines the algorithm listing environment
{   
    \refstepcounter{nalg} %increments algorithm number
    \captionsetup{labelformat=algocaption,labelsep=colon} %defines the caption setup for: it ises label format as the declared caption label above and makes label and caption text to be separated by a ':'
    \lstset{ %this is the stype
        float,
        floatplacement=H,
        mathescape=true,
        frame=tB,
        numbers=left, 
        numberstyle=\scriptsize,
        basicstyle=\footnotesize, 
        keywordstyle=\color{black}\bfseries\em,
        keywords={,input, output, and, or, if, then, else, foreach, in, do, begin, end, } %add the keywords you want, or load a language as Rubens explains in his comment above.
        morecomment=[l]{//}, % l is for line comment
        commentstyle=\itshape\color{green},
        numbers=left,
        xleftmargin=.04\textwidth,
        #1 % this is to add specific settings to an usage of this environment (for instnce, the caption and referable label)
    }
}
{}

%% Some custom commands:
\newcommand{\ket}[1]{\vert #1 \rangle}

%% Information about the title, etc.
\title{Towards Distributed Quantum Algorithms}
\author{Pablo Andres-Martinez}

%% If the year of submission is not the current year, uncomment this line and 
%% specify it here:
% \submityear{1785}

%% Optionally, specify the graduation month and year:
% \graduationdate{February 1786}

%% Specify the abstract here.
\abstract{%
}

%% Now we start with the actual document.
\begin{document}

\tikzset{
  pics/box/.style args={#1/#2/#3/#4/#5/#6/#7/#8}{
    code = {
      \coordinate[name prefix .., below left=#2 and -#3 of circuit.north west] (#1coord);
      \node[rounded corners, fill=black, opacity=0.2, minimum height=#4,minimum width=#5] at (coord) (theBox) {};
      \node[below right=#7 and #8 of theBox.north west, font=\scriptsize\itshape] (text) {#6};
    }
  },
  pics/opaquebox/.style args={#1/#2/#3/#4/#5/#6}{
    code = {
      \coordinate[name prefix .., below left=#2 and -#3 of circuit.north west] (#1coord);
      \node[rounded corners, font=\small\itshape, fill=black!20, opacity=1, minimum height=#4,minimum width=#5] at (coord) (theBox) {#6};
    }
  },
  pics/cut/.style args={#1/#2}{
    code = {
      \draw[name prefix .., dash pattern={on 7pt off 2pt on 1pt off 2pt}, line width=1pt, opacity=0.65] (#1) -- (#2);
    }
  },
  pics/ebit/.style args={#1/#2/#3}{
    code = {
      \node[name prefix .., below left=#2 and -#3 of circuit.north west, rectangle, fill=white, minimum height=10mm, minimum width=#3] (#1clear) {};
      \coordinate[below left=1.5mm and 0mm of clear.north east] (one);
      \coordinate[below=7.04mm of one] (two);
      \draw[name prefix ..] (#1one) edge[out=180,in=180,looseness=1.5] (#1two);
    }
  },
  pics/ebitLessClear/.style args={#1/#2/#3}{
    code = {
      \node[name prefix .., below left=#2 and -#3 of circuit.north west, rectangle, fill=white, minimum height=10mm, minimum width=5mm] (#1clear) {};
      \coordinate[below left=1.5mm and 0mm of clear.north east] (one);
      \coordinate[below=7.04mm of one] (two);
      \draw[name prefix ..] (#1one) edge[out=180,in=180,looseness=1.5] (#1two);
    }
  },
  pics/ebitLong/.style args={#1/#2/#3}{
    code = {
      \node[name prefix .., below left=#2 and -#3 of circuit.north west, rectangle, fill=white, minimum height=25mm, minimum width=#3] (#1clear) {};
      \coordinate[below left=1.5mm and 0mm of clear.north east] (one);
      \coordinate[below=21.12mm of one] (two);
      \draw[name prefix ..] (#1one) edge[out=180,in=180,looseness=1.5] (#1two);
    }
  },
}

%% First, the preliminary pages
\begin{preliminary}

%% This creates the title page
\maketitle

%% Acknowledgements
\begin{acknowledgements}
\end{acknowledgements}

%% Next we need to have the declaration.
\standarddeclaration

%% Finally, a dedication (this is optional -- uncomment the following line if
%% you want one).
% \dedication{To my mummy.}

%% Create the table of contents
\tableofcontents

%% If you want a list of figures or tables, uncomment the appropriate line(s)
% \listoffigures
% \listoftables

\end{preliminary}

\chapter{Introduction}
\label{chap:Introduction}

In this thesis, we propose an algorithm for the automated translation of programs from monolithic quantum computers to distributed grids of quantum computers. The reader is not required to have any prior knowledge of quantum computing, as we will introduce the concepts needed in this thesis in an incremental manner throughout the thesis. In some occasions, we will use analogies with classical computer science -- ranging from distributed architectures to compilers -- that should help understand the key concepts of distributed quantum computing. Our main contribution consists in reducing the problem of efficiently distributing a quantum circuit to the problem of hypergraph partitioning, which is a problem thoroughly studied in the literature of classical computer science.

Quantum computing aims to take advantage of quantum mechanics to speed-up computations. There are many examples of problems that, although solvable by a classical computer, the time it would take to compute them is unreasonable in practice, regardless how large or fast your classical computer is. Some of these \textit{intractable problems} can be solved efficiently on a quantum computer. Well-known examples of such problems are:

\begin{itemize} 
\item \textit{Factorisation of large numbers}: In classical computers, all known factorisation algorithms take exponential time with respect to the input size. It is strongly believed that there is no way a classical computer can solve the problem efficiently -- in fact, we are so confident about it that most widely used encryption systems, like RSA, rely on this. However, quantum computers are capable of solving that problem in polynomial time (i.e.\ making it tractable), using Shor's algorithm~\citep{Shor}.
\item \textit{Unstructured search}: The aim is to perform a brute-force search (i.e.\ requiring no prior knowledge about the search space) over \(N\) data-points. Classical computers have no other option than testing each data-point, so the time they take to perform the search is proportional to \(N\). With a quantum computer, using Grover's algorithm~\citep{Grover}, the search is done in time proportional to \(\sqrt{N}\).
\end{itemize}

Besides, a recent result by \citet{BQPSepPH} gives formal proof of the existence of a large family of problems that a classical computer may never solve in polynomial time, but are solvable in polynomial time on a quantum computer. Nevertheless, all of these results are theoretical in nature, and giving experimental evidence of this gap in computational power is a highly active area of research, known as \textit{quantum supremacy}. 

Quantum computing would be very valuable in many areas of research that deal with problems that are intractable on classical computers. Some of the main applications that have been discussed in the literature are:

\begin{itemize}
\item \textit{Chemistry, medicine and material sciences}: Calculating molecular properties on complex systems is extremely demanding for classical computers. However, polynomial algorithms for this kind of problems are known for quantum computers~\citep{TowardsQuantumChemistry}. %Quantum computers are likely to trigger a revolution on areas of science that need to model molecules and their interactions.
\item \textit{Machine learning}: Finding patterns in a large pool of data is the essence of machine learning. Multiple quantum algorithms have been shown to be able to detect patterns that are believed not to be efficiently attainable classically~\citep{QuantumMachineLearning}.
\item \textit{Engineering}: Optimisation and search problems are common in almost every area of engineering. Quantum computers are particularly well suited for these tasks, with Grover's algorithm (unstructured search) being an obvious example.
\end{itemize}

For any of these applications we will require large scale quantum computers. Due to the obstacles in the way of building a large quantum computer, some experts have advocated the alternative of building a \textit{quantum multicomputer}: a grid of small quantum computers that cooperate to perform an overall computation~\citep{DistributedQCHW}. Small quantum computers have already been built, and the prototype of the technology needed to connect them in a grid is known and attainable. Although there is still a long way until we can build a distributed grid of quantum computers, the technology required may be developed before large scale monolithic computers become feasible. And even if this is not the case, distributed quantum computing would still be a desirable asset, for similar reasons it is for classical computing: a cluster of small computers may be cheaper than a large mainframe, and a multicomputer is intrinsically modular, so we may easily adjust the amount of resources we dedicate to a computation.

Unfortunately, there is virtually no support for developing programs to be run on distributed quantum architectures. The literature on quantum algorithms and quantum programming languages focuses almost exclusively on a single model of quantum computation -- the circuit model --, meant to be run on a monolithic computer. In this thesis, we intend to make a first step towards the spread of distributed quantum algorithms, providing an algorithm that distributes any input circuit across an arbitrary number of quantum computers.

\section{Outline of chapters}

Together, Chapters~\ref{chap:Overview} and~\ref{chap:Distributed} correspond to the literature review relevant to this thesis, and they identify the problem we intend to solve. Chapter~\ref{chap:Overview} introduces the key concepts of quantum computer science that will be required throughout this thesis. Most importantly, \S\ref{Principles} explains the principles of quantum computing, and gives an intuition behind the speed-up it achieves; while \S\ref{Hardware} discusses the scalability challenges of quantum computers, and presents the main model of quantum computing -- the circuit model. Chapter~\ref{chap:Distributed} gives an introduction to distributed quantum computing: In \S\ref{Ebits} we explain how quantum information may be communicated across QPUs, \S\ref{IntroDistributing} reviews the main paper our thesis is built upon~\citep{NonLocalCNOT}, and \S\ref{DQC_Architecture} describes an abstract distributed quantum architecture.

Our contributions are presented in Chapter~\ref{chap:Project}. First, in \S\ref{NonLocalGates} we propose two extensions of the work by \citet{NonLocalCNOT}. \S\ref{EfficientDistrib} is the core of the thesis, where we describe our algorithm for automated distribution of quantum circuits. We present our first version of the algorithm, then we describe its two extensions, corresponding to those we proposed in \S\ref{NonLocalGates}. \S\ref{Interchange} outlines a third potential extension that we propose as further work.

Finally, in Chapter~\ref{chap:Results}, we show the results of using our algorithm to distribute different benchmark circuits drawn from the literature. We discuss these results and study the effect of its two extensions. Chapter~\ref{chap:Conclusions} ends this thesis by drawing some conclusions and suggesting further work.
\chapter{Quantum Computing: A brief overview}

Quantum computing aims to take advantage of quantum mechanics to speed-up computations. There is strong theoretical evidence that quantum computers are capable of solving some problems substantially faster than standard (classical) computers. Well known examples are:

\begin{itemize} 
  \item Shor's algorithm for \textit{polynomial-time} large number factorisation~\citep{Shor}. There is no known algorithm on classical computers that can perform this task in polynomial time, and it is suspected not to be possible\footnote{RSA, a widely used encryption system, relies its security on the assumption that factorisation of large numbers can not be computed efficiently.}. Quantum computers theoretically provide an exponential speed-up on this task.
  \item Grover's algorithm for efficient unstructured search~\citep{Grover}. The algorithm performs a brute-force search (i.e.\ requiring no knowledge about the search space) over a \(N\) data-points in time proportional to \(\sqrt{N}\). A brute-force search in a classical computer should always take time proportional to \(N\).
\end{itemize}

Besides, in May of the present year, Raz and Tal~\citep{BQPSepPH} gave formal proof of the existence of problems that a classical computer may never solve in polynomial time, but are solvable in polynomial time on a quantum one. Nevertheless, all of these results are theoretical in nature and there are some caveats on their practical implementation, discussed in \S~\ref{Challenges}. Giving experimental evidence of this time-efficiency separation between quantum and classical computers is a highly active area of research, known as \textit{quantum supremacy}. 

Quantum computing would be very valuable in many areas of research where classical computers are unable to solve problems efficiently. Some of the main applications that have been discussed in the literature are:

\begin{itemize}
  \item \textit{Chemistry, medicine and material sciences}: Calculating molecular properties on complex systems is an intractable problem for classical computers. However, polynomial algorithms for this problem are known for quantum computers~\citep{TowardsQuantumChemistry}. Hence, quantum computers are likely to trigger a revolution on areas of science that need to model molecules and their interactions.
  \item Machine learning: Finding patterns in a large pool of data is the essence of machine learning. Multiple quantum algorithms have been shown to be able to detect patterns that are believed not to be efficiently attainable classically~\citep{QuantumMachineLearning}.
  \item Engineering: Optimization and search problems are common in almost every area of engineering. Quantum computers are particularly well suited for these tasks, with Grover's algorithm being a clear example.
\end{itemize}

For any of these applications we will require large scale quantum computers. Due to the obstacles in the way of building a large mainframe quantum computer (see \S~\ref{Challenges}), some authors have advocated the alternative of building a quantum multicomputer: a grid of smaller quantum computer units that cooperate in performing an overall computation~\citep{DistributedQCHW}. In the present work, particularly in Chapter~\ref{Project}, we contribute to this perspective, providing a method for efficiently distributing any quantum program originally designed for a monolithic quantum computer.

\section{The principles of Quantum Computing}
\label{Principles}

The advantage of using quantum mechanics to perform computations is usually traced down to the following three principles:

\begin{itemize}

  \item \textit{Superposition}: In classical computing, the unit of information is the \textit{bit}, which may take one of two values: \(0\) or \(1\). In quantum computing, the \textit{bit}'s counterpart is the \textit{qubit}, whose value may be \textit{any linear combination} of \(0\) and \(1\), known as a \textit{superposition}, and usually written as: \[\ket{qubit} = \alpha\ket{0} + \beta\ket{1}\] where \(\alpha\) and \(\beta\) are complex numbers that must satisfy: \(\alpha^2 + \beta^2 = 1\).

  A popular analogy of a qubit's superposition is a coin spinning\footnote{Note this is just an analogy, and while a coin spinning can be perfectly modelled using classical physics, a qubit can not. In fact, superposition is key in the (even weirder) two other principles that differentiate quantum and classical computing.
  }: the classical states (\(0\) and \(1\)) are \textit{heads} and \textit{tails}, but when the coin is spinning, its state is neither of them. If we knew exactly how the coin was set spinning, we would be able to describe the probability distribution of seeing heads or tails when it stops; these would be our \(\alpha^2\) and \(\beta^2\) values. We may \textit{measure} a qubit, and doing so corresponds in our analogy to stopping the coin: we will get either \(0\) or \(1\) as outcome.

  The essential aspect of this analogy is that, before measurement, the \textit{qubit}'s state is neither \(0\) nor \(1\). Through certain operations (that would correspond to altering the axis of spin of the coin), we may change the coefficients \(\alpha\) and \(\beta\) of the superposition. Interestingly, in quantum computing, we encode input and read output (after measurement) as standard classical binary strings, and thus, \textit{for input/output we use as many qubits as bits would be required}. What superposition provides is the ability to -- during mid-computation -- maintain a superposition of all potential solutions to the problem, and update all of them simultaneously with a single operation to the qubits. In some sense, superposition allows us to explore multiple choices/paths of the computation, using only the resources required to explore a single one of those paths. And the number of path we can explore simulatenously can be up to exponential in comparison to the classical case, as a string of \(N\) qubits may be in a state of superposition of all the \(2^N\) classical states. This is the reason behind the exponential speed-up of Shor's algorithm.

  \item \textit{Interference}: As we just discussed, superposition gives us the ability to simultaneously explore different paths to solve a problem. However, in the end we will need to measure the qubits -- stop the coins, in order to read heads or tails -- and the result will be intrinsically random. For quantum computing to be any better than a probabilistic classical computer, we require the ability to prune the paths that have led to a dead-end. This is precisely what \textit{interference} provides: some operations on the qubits may make different classical states in the superposition cancel each other out. Interference is at the core of any speed-up achieved by a quantum algorithm, and taking advantage of it is the main challenge when designing quantum algorithms.

  \item \textit{Entanglement}: Quantum mechanics allows us to have a pair of qubits \(a\) and \(b\) in a superposition such as: \[\ket{a,b} = \frac{1}{\sqrt{2}}\ket{0,0} + \frac{1}{\sqrt{2}}\ket{1,1}\] This implies that, when we measure the qubits, we may only read \(a=0, b=0\) or \(a=1, b=1\) as outcome (the coefficients for \(\ket{0,1}\) and \(\ket{1,0}\)) are both \(0\)). Then, what happens if we only measure \(a\)? In this case, we would also know \(b\)'s outcome, without measuring it. More surprisingly, if we measured both \(a\) and \(b\) at the same instant, we would obtain \(a=0, b=0\) half of the times and \(a=1, b=1\) the other half. In short, it seems like acting on one qubit has an instantaneous effect on the other. Whenever a group of qubits exhibits this property, we say they are \textit{entangled}. Entanglement holds regardless how far apart \(a\) is from \(b\); for instance, they could be on two different quantum processing units of a distributed grid. Indeed, entanglement will be key in our discussion of distributed quantum algorithms, and we explain how to use it to perform non-local operations in \S~\ref{IntroDistributing}.

\end{itemize}


\section{Building Quantum Computers}
\label{Hardware}

Physicists have come up with different ways of realising qubits in labs. The key idea is to find a physical system that displays non-classical behaviour, and put it under the appropriate circumstances so we can manage its quantum properties, but noise in the environment may not interfere with these. \citet{ArchitectureSurvet} gives an excellent survey of the state of the art of quantum architectures. Among them, the three closest to experimental realisation are:

\begin{itemize}

  \item \textit{Optics}: The state of a qubit is represented in the properties of photons, for instance, their polarization~\citep{OpticsQC}. A great advantage of this technology is that photons can be easily sent over long distances, while preserving the quantum state. Thus, protocols in quantum information that heavily rely on communication, such as Quantum Key Distribution~\citep{QKD}, are usually discussed and experimented using optics. The downside of optics is that it is very difficult to make photons interact, which is required for other than single qubit operations.

  \item \textit{Ion-traps}: Each qubit is embodied as an ion, confined inside a chamber by means of an electric or magnetic fields. The qubit is acted upon by hitting the ion with electromagnetic pulses (e.g.\ laser light or microwave radiation). This is one of the most promising technologies for the future of quantum computing, with groups of experimentalists having proposed how to scale up the technology~\citep{HensingerIonTraps}.

  \item \textit{Superconductors}: Small circuits, similar to classical electrical circuits, are cooled down to near absolute zero so the quantum interactions of electrons are not obscured by other perturbations. Then, different parts of the circuit encode different qubits, which can be acted upon by applying different electric potentials. One of the main advantages of this technology is that the challenges for scaling up such circuits, apart from the cooling system, are similar to the challenges we have encountered over the years for classical computers. Therefore, this technology seems to be the most feasible in the near future and evidence of that is the fact that, using it, both IBM and Intel have already built small generic-purpose quantum computers of 17-20 qubits.

\end{itemize}

However, for quantum computers to be useful in real world applications, their qubit count should raise up, at the very least, one order of magnitude. And, unfortunately, increasing the amount of qubits in a quantum computer is particularly difficult, due to some caveats we will now discuss.


\subsection{Scalability challenges}
\label{Challenges}

There are two main challenges to overcome in order to build large scale quantum computers:

\begin{itemize}

  \item \textit{Decoherence}: In \S~\ref{Principles} we discussed the importance of having superposition in quantum computing, and we compared a qubit in superposition with a coin spinning. For the same reason why a coin spinning will eventually stop, a qubit in superposition will eventually degenerate into a classical state (i.e.\ either \(\ket{0}\) or \(\ket{1}\)): physical systems have a tendency towards the state at which they are most stable, for the coin it is laying flat, for the qubit it is losing its superposition. This phenomenon is known as \textit{decoherence} and it will always occur in any given technology\footnote{A revolutionary technology, \textit{anyonic} (a.k.a.\ topological) quantum computing, has been proposed to theoretically avoid the problem of decoherence by using physical systems that can be completely protected against it~\citep{Anyonic}. Although promising, currently this proposal has little experimental underpinning, and it is not regarded as attainable in the near future.}, in some faster than others. Experimentalists attempt to increase the time it takes for the state of the qubit to degenerate, which in both ion-trap and superconductor technologies it is in the order of microseconds. Decoherence constitutes the main constraint to scalability of quantum computers, as it dictates the lifespan of qubits, limiting the number of operations that can be applied in a single program. 

  Certainly, the state of bits also degenerates in classical computers. However, in their case this is easier to account for: intuitively, we can keep monitoring the bits, and make sure to correct any unwanted change. This is not so simple in quantum computers, as monitoring a qubit would require \textit{measuring} it, and that destroys any quantum superposition. Nevertheless, it is still to some extent possible to protect our quantum state from errors -- either due to decoherence or imperfect hardware -- through quantum error correction routines~\citep{QuantumErrorCorrection}. This is a very active area of research, and it will be key for the implementation of reliable large scale quantum computers.


  \item \textit{Connectivity}: In order to run complex computations on qubits, we will need to be able to apply apply multi-qubit operations on any subset of the available qubits. However, it is not realistic to expect that quantum computers will have fast connectivity between all qubits, due to spatial separation of these in the hardware. In classical systems, this problematic is solved by a memory hierarchy, with a ceaseless flow of data going up and down of it, from main memory to registers and back. However, the memory hierarchy model works because most of the data can stay idly in main memory while computation on the registers data is carried out. In quantum computers, we must avoid qubits being idle, as decoherence prevents the existence long-lasting memory. An alternative found in classical computers is to distribute the computation across different processing units, each having its own local memory which they use intensively, and communicating -- through message passing -- as little as possible. In \S~\ref{DQC_Architecture}, we discuss an abstract distributed quantum architecture in detail.

\end{itemize}


\subsection{Models of computation}
\label{Models}

In this section, we give a brief introduction to some models of quantum computation relevant to the this thesis.

\begin{itemize}

  \item \textit{Circuit model}: Also known as the network model. Any operation on \(n\) qubits -- as long as measurement (i.e.\ destruction of information) is not involved -- can be represented as square matrix on complex numbers, of dimension \(2^n\). These matrices are always unitary, which means that a matrix \(U\)satisfies \(UU^\dag = I = U^\dag U\), where \(I\) is the identity matrix and \(A^\dag\) is the conjugate transpose of \(A\). Essentially, unitarity ensures that any operation on qubits can be reversed (i.e.\ undone), reason why this model is sometimes called the reversible model. Multiplying matrices \(AB\) corresponds to applying the operation described by \(B\) first, then \(A\), on the same qubits. Application of two operations on disjoint set of qubits corresponds to the Kronecker product of the matrices \(A \otimes B\). The fundamental concept is that any matrix can be represented as a product of other matrices, so we may decompose any operation into smaller building blocks: quantum gates.

  Qubits are pictured as wires to which quantum gates are applied, similarly to a classical digital circuit. The set of quantum gates used is dependent on the architecture. There exist an (uncountable) infinite amount of different quantum operations, but a small finite set of them is enough to approximate any of them, up to a desired error factor. The most common choice of such a universal gate-set is \texttt{Clifford+T}, which contains six one-qubit gates, and a single two-qubit gate. The depiction of such gates and some of their most important properties are shown in Figures~\textbf{TODO}. Circuits are read from left to right.

  \textbf{TODO:} A figure depicting, and giving the matrix of, each of CNOT, X, Y, Z, H, S and T. Figures showing HXH = Z; XX = I, ZZ = I, YY = I, HH = I; SS = Z; TT = S; H2 CNOT H2 = NOTC, and also their algebraic notation.

  The CNOT gate (Figure~\ref{fig:CNOT}) is particularly interesting. The qubit where the filled dot is acts as the `control', and the qubit with \(\oplus\) acts as `target'. Whenever the control is \(\ket{0}\), the target is unaffected; but if it is \(\ket{1}\), an X gate (Figure~\ref{fig:X}) is applied, flipping the value of the qubit. This works in any superposition, so if in \[\ket{c,t} = \alpha\ket{0,0} + \beta\ket{0,1} + \gamma\ket{1,0} + \delta\ket{1,1}\] \(\ket{c}\) were acting as control and \(\ket{t}\) as target, the outcome would be: \[CNOT \ket{c,t} = \alpha\ket{0,0} + \beta\ket{0,1} + \gamma\ket{1,1} + \delta\ket{1,0}\]

  \textbf{TODO:} A figure showing an abstract quantum circuit.

  \item \textit{MBQC model}: Initials stand for Measurement Based Quantum Computing. Unlike the circuit model, where measurements are done at the very end of the circuit, MBQC carries out computations by means of repeatedly measuring an initial entangled resource. The process, sketched in Figure~\ref{fig:MBQC} can be thought of as sculpting a statue from a block of granite. The initial entangled resource, which is a collection of entangled qubits forming a lattice structure, corresponds to the granite block. By measuring some qubits in the lattice -- hitting the rock with a chisel -- we remove some of the excess qubits, changing the overall state in the process. The outcome of measurements is probabilistic so, in order to provide deterministic computation, we must apply corrections on the neighbouring qubits whenever the measurement outcome deviated from the desired result. After multiple iterations of measurements and corrections, we end up with a set of qubits encoding the result. The input was incorporated into the lattice at the beginning of the process.

  \textbf{TODO:} Figure from slides %\label{fig:MBQC}

  In this way, any computation may be performed by applying 1-qubit measurement and 1-qubit correcting gates (controlled by classical signals). The initial resource state contains all the entanglement that is required, which may be prepared experimentally through multi-qubit interactions, such as Ising interactions~\citep{1WQC}, which are within our experimental capabilities. Hence, in some sense this model solves the problem of connectivity by applying a single large operation at the beginning of the process, and then only requiring cheap single qubit operations. The main drawback of MBQC is the large amount of qubits that are required for even the simplest of operations, but given this might be surmountable considering the rest of the architecture is greatly simplified. The MBQC model was presented for the first time by Raussendorf and Briegel~\citep{1WQC} under the name of \textit{one-way quantum computer}, highlighting its main difference with the circuit (reversible) approach. 

  \item \textit{Distributed model}: We may find a balance between the circuit model and MBQC, where multiple small quantum processing units run fragments of the overall circuit, and communication is achieved through a shared entangled resource. This model has been discussed in detail in the literature~\citep{DistributedQCHW} and it is at the core of the Networked Quantum Information Technologies Hub (NQIT)\footnote{a project supported by the UK National Quantum Technology program, aiming to provide scalable quantum computing}. In \S~\ref{DQC_Architecture}, we discuss an abstract distributed quantum architecture in detail.

\end{itemize}


\section{Programming on Quantum Computers}

As of today, most quantum programming languages are high level circuit descriptors: they provide the means to define circuits gate by gate, or build them up from combinations of smaller circuits. In this category fall all the well-known languages, such as \textit{QCL}~\citep{QCL} (imperative paradigm, and one of the first quantum programming languages ever implemented), \textit{Q\#}~\citep{QLang} (imperative, designed by Microsoft), and \textit{Quipper}~\citep{Quipper} (functional, built on top of Haskell). Besides, there are attempts at designing quantum programming languages that are completely hardware agnostic, meaning they aim to describe the computation, rather than a particular circuit that implements it. Examples of these are the different attempts at defining a quantum lambda calculus, for instance van Tonder's~\citep{VanTonder} or Diaz-Caro's~\citep{Diaz-Caro}). However, these are still early in their development and tend to be particularly verbose.

Additionally, most of the literature on quantum algorithms describes these by explicitly giving circuits that implement them. There is a constructive procedure, given by the Solovay-Kitaev theorem~\citep{SolovayKitaev}, that takes any circuit and a choice of universal gate-set and outputs an efficient equivalent circuit using only those gates. Hence, programmers do not need to worry about the gates they are using when describing their circuits.

Unfortunately, the fact that algorithms are almost exclusively defined in the circuit model implies that other models of quantum computing (introduced in \S~\ref{Models}) are disregarded by a large portion of the community. In order to make other models of computation accessible, we need to provide automated procedures for transforming algorithms from the circuit models to these (and vice versa). Work has been done on the transformation from circuit to MBQC and backwards, the latter being the most challenging~\citep{gflow}. However, there is little amount of literature describing how to go from the circuit model to the distributed model. In \S~\ref{IntroDistributing} we give an overview of the existent work on that aspect, and identify the gap on the literature we aim to answer in this thesis.


\section{Summary}

As a wrap up, here are the key concepts to keep in mind while reading the rest of this thesis:

\begin{itemize}
  \item Quantum computers provide a computing power well beyond the capabilities of classical computers, which would be exploitable in many areas of science.
  \item Small quantum computers are already available. 
  \item Scaling up is a challenging problem due to: \textit{decoherence}, which may be overcome by the joint effort of error-correction, physics and engineering communities; and \textit{connectivity}, which may be solved using distributed architectures.
  \item There is practically no programming support for distributed architectures.
\end{itemize}
\chapter{Distributed Quantum Computing}
\label{chap:Distributed}

As we discussed in \S\ref{Hardware}, there are different approaches on how to build quantum computers. Now that many of these have been experimentally demonstrated, having built small quantum computers, the question of how to scale up is increasingly relevant. This has led to the proposal of distributed architectures~\citep{ArchitectureSurvey}.

In classical computing, an standard example of a distributed computer is the Non-Uniform Memory Access (NUMA) architecture: A system of independent computing nodes, each having its own local memory. In order to collaborate to perform an overall computation, the different nodes will need to communicate. In NUMA, they do so by accessing each other's memory. While nodes can manage their own local memory efficiently, accessing another node's memory is slow. Hence, we always attempt to minimise the amount of communication between nodes. A distributed quantum computer would follow the same principles, where each quantum processing unit (QPU) would own a collection of qubits (its local memory) and may access another QPU's qubits at the cost of some overhead, using entanglement.

\section{Communication through entanglement}
\label{Ebits}

For a QPU to be able to access another's qubit, we must provide them with some sort of communication channel. Simply using a classical channel (sending bits) is not helpful: the whole point of representing the state of the computation in qubits is that they may be in a superposition of classical states, which would take up to an exponential amount of space and processing on classical bits. We could consider physically transporting the system that encodes the qubit from one QPU to another, and while that is certainly possible with photons, in general it is not feasible to have a channel that is both fast and protects well against information loss due to decoherence.

In \S\ref{Principles}, we explained it was possible to affect a distant qubit by acting on another qubit with which it was entangled. We wish to exploit this property in order to allow a QPU to query another's QPU qubit. There are different levels of how strong a pair of qubits is entangled -- intuitively, how much they affect each other. This is often formalised as the correlation between the qubits possible measurement outcomes; for instance, the pair of qubits \(\frac{1}{\sqrt{2}}\ket{0,0} + \frac{1}{\sqrt{2}}\ket{1,1}\) is said to be \textit{maximally entangled}, as the possible measurement outcomes are exclusively either \(\ket{0,0}\) or \(\ket{1,1}\), always matching in both qubits\footnote{\, In total, there are four \textit{maximally entangled} states of a pair of qubits: \(\frac{1}{\sqrt{2}}\ket{0,0} + \frac{1}{\sqrt{2}}\ket{1,1}\) and \(\frac{1}{\sqrt{2}}\ket{0,0} - \frac{1}{\sqrt{2}}\ket{1,1}\) give perfect correlation, while \(\frac{1}{\sqrt{2}}\ket{0,1} + \frac{1}{\sqrt{2}}\ket{1,0}\) and \(\frac{1}{\sqrt{2}}\ket{0,1} - \frac{1}{\sqrt{2}}\ket{1,0}\) give perfect anti-correlation}. Naturally, the most efficient communication channel will take advantage of entanglement in its strongest form, and so we will make use pairs of qubits entangled in this particular maximally entangled state. This qubit pair configuration is generally known as a Bell state, and Figure~\ref{fig:bell} shows how to prepare it.

\textbf{TODO}: fig:bell, show the circuit to make a bell state, how it is represented usually as a wire C and its Diract state.

An interesting property of the Bell state is shown in Figure~\ref{fig:sliding}: if a quantum gate, whose matrix representation is symmetric, is applied to one of the qubits, it is the same as if the gate was applied to the other qubit. In some sense, the gate can `slide' through the entanglement, like beads on a string; as if the entangled state were a curved wire, connecting the pair of qubits. Hopefully, this serves as a first intuition of how Bell states are a natural choice for implementing quantum comunication.

\textbf{TODO}: fig:sliding, show how a H gate can slide through an ebit.

So far, we have explained how to generate a Bell state inside a QPU (as in Figure~\ref{fig:bell}). However, what we aim for is that two different QPUs each own one of the qubits from the Bell state. The challenge is then to send one of the qubits to another QPU, while preserving the entangled state. The problem of sharing a Bell state between two parties is solved by the \textit{entanglement distillation} protocol~\citep{DistillationProtocol}, which ensures the Bell pair is transmitted to their destination with an arbitrarily small error factor. We will now give a brief description of this protocol.

\subsection{Entanglement distillation}
\label{Distillation}

\textbf{TODO}: This section

Sending an arbitrary quantum state across a channel is difficult, we could do it either fast or reliably, but not both. But if you do know what you are sending, you just send a lot of them.

Non-fidelity. If the ebit is not exactly in a state of the form \(\alpha\ket{0,0} + \beta\ket{1,1}\) (i.e.\ it has a non-zero coefficient for \(\ket{0,1}\) or \(\ket{1,0}\))

Often in distributed quantum computing literature, a Bell state shared by two QPUs is known as an \textit{ebit} (entangled-bit). We will also use this convention, and refer to each of the two qubits in the Bell state as the two \textit{halves} of the ebit.

\section{Distributing circuits}
\label{IntroDistributing}

We now explain how ebits are used to allow a QPU to peek into another's qubits. Here, we will introduce the proposal of \citet{NonLocalCNOT}. Later on, in \S\ref{NonLocalGates}, we will extend this work with our own contributions. 

We aim to split a given circuit and distribute the fragments across multiple QPUs. The gates that should operate over qubits on different QPUs are known as \textit{non-local} gates. As we previously mentioned, any circuit can be converted to Clifford+T circuit (thanks to the Solovay-Kitaev theorem). In Clifford+T, the only gate that operates on more than one qubit is the CNOT. Hence, we only need to understand how CNOTs can be implemented non-locally.

\textbf{TODO:} A figure with a simple circuit and dashed lines identifying how we want to split it, and highlighting a non-local CNOT.

The construction we will use is a slight variation of what was proposed by \citet{NonLocalCNOT}, and it is shown in Figures~\textbf{TODO}. In principle, we will use an \textit{ebit} per non-local CNOT. We will call the QPU that holds the target qubit (the one with a \(\oplus\)) the `target QPU' and similarly for the control qubit. 

\textbf{TODO}: showing the overall scheme, with cat-entangler and cat-disentangler as black boxes.

The implementation of a local CNOT gate has three steps. First, we must apply what the authors refer to as the \textit{cat-entangler} (Figure~\ref{fig:cat-entangler}), which creates a local `copy'\footnote{\, Note that there is no such thing as `copying' a quantum state (due to the non-cloning theorem). What we mean here by `copying' is generating, from \(\ket{\psi} = \alpha\ket{0} + \beta\ket{1}\), the state \(\ket{\psi} = \alpha\ket{0,0} + \beta\ket{1,1}\) which is fundamentally different from an actual copy: \(\ket{\psi,\psi} = \alpha^2\ket{0,0} + \alpha\beta\ket{0,1} + \alpha\beta\ket{1,0} + \beta^2\ket{1,1}\).} of the control qubit inside the target QPU. In the process, the ebit half in the control QPU is measured (and thus destroyed), and the outcome is used to correct the other half, in the same spirit as in the MBQC model. Notice that the only information physically crossing the boundary between blocks is the \textit{classical} outcome of the measurement (a bit, either \(0\) or \(1\)).

\textbf{TODO}: Show the circuit for the cat-entangler and the resulting state

Then, the CNOT gate is \textit{applied locally} inside the target QPU, between its ebit half and the target qubit. At the end, the \textit{cat-disentangler} must be applied (Figure~\ref{fig:cat-disentangler}), which simply destroys -- with a measurement -- the remaining ebit half and then corrects the control qubit, so the randomness of the measurement is counteracted. Once again, only classical information crosses the boundary.

\textbf{TODO}: Show the circuit for the cat-disentangler and the resulting state

In this way, we have implemented a non-local CNOT gate using one ebit and two classical bit messages between QPUs. However, the true advantage of this approach is attained when multiple non-local CNOTs are implementing using a single ebit: Once the cat-entangler is applied, any number of CNOTs whose target is in the same QPU, and that are controlled by the same qubit, may all be implemented by using the same ebit half as control, as shown in Figure~\ref{fig:multipleCNOTs}.

\textbf{TODO}: Show the circuit for many CNOTs, now with cat-entangler and everything

Now, depending of how we choose to partition the circuit, there will be different groups of CNOTs that we may be able to implement using a single ebit. We will then wish to find the partition that requires the fewest ebits to implement all of its CNOT gates. This optimization problem is not discussed in the original paper, nor in any other work, as far as we know. It will be our main contribution in this thesis, along with an extension of the results just explained, both found in Chapter~\ref{chap:Project}.


\section{Distributed quantum architectures}
\label{DQC_Architecture} 

A distributed quantum computer will have multiple quantum processing units (QPUs), each managing a small collection of qubits as their local memory. It should also have a subsystem specialised in generating and sharing ebits. As we discussed in \S\ref{Distillation}, there is a compromise between the quality of the ebit -- the quality of the communication channel -- and the time it takes to prepare it. Fortunately, \citet{NoisyChannels} showed that efficient distributed quantum computation using noisy ebits is feasible. Still, the generation of \textit{ebits} remains the main bottleneck of the architecture, and thus we will want to minimise the required ebit count as much as possible. 

\textbf{TODO} Detail the abstract distributed quantum architecture.

\textbf{TODO} Authors such as \citet{DistributedQCHW} have discussed the experimental construction of a similar architecture. It is a multicomputer that uses photons for communication across QPUs, entangling stuff. 
\chapter{Automated Distribution of Quantum Algorithms}
\label{chap:Project}

\section{Implementing non-local CNOTs}
\label{NonLocalGates}

In \S~\ref{IntroDistributing}, we explained the proposal by~\citet{NonLocalCNOT} of how to implent a non-local CNOT. We will now extend their results.

The first thing to notice is that the CNOTs need not immediately follow one on the control qubit. Assuming the circuit uses only the Clifford+T gate-set, all the 1-qubit gates but H can be pushed through the CNOT, and we may transform a circuit so the controls of different CNOTs are indeed together. The way they are pushed through is shown in Figures~\textbf{TODO}. All of these can be checked by calculating the corresponding matrices and seeing they match.

\textbf{TODO}: Figures of how to push gates through control.

The second improvement comes by realising that the trick used to implement multiple CNOTs controlled by the same qubit can also be applied if multiple CNOTs target the same qubit. The derivation is shown in Figure~\ref{CNOTtargetProof}, which uses some of the properties listed in \S~\ref{Models}.

\textbf{TODO}: CNOTtargetProof

Finally, we will be interested in also pushing gates through the target of a CNOT, in order to bunch them up together too. Doing so is slightly more elaborate, and the relevant rules are derived in Figures~\textbf{TODO}.

\textbf{TODO}: Derivation of pushing gates through target (playing with Hadamard).



\section{Finding the best partition}

In this section we explain how we search for a suitable partition of the circuit. As discussed in \S~\ref{DQC_Architecture}, we will be interested in achieving the minimal count of required ebits possible, while maintaining a balance of the number of qubits in each block. The problem is very similar to the \((k,\varepsilon\) graph partitioning problem, where a graph partitioning in \(k\) subgraphs has to be found, minimising the number of edges that have their incident vertices in different blocks, and ensuring the number of vertices in each block is within the \(\varepsilon\) tolerance factor: \((1 \pm \varepsilon)\frac{N}{k}\), where \(N\) is the total number of vertices in the graph. Essentially, the qubits of the circuit would become vertices of a graph, whose edges would correspond to each CNOT in the circuit. Whenever an edge is `cut' (i.e.\ its vertices are in different subgraphs), the CNOT it represents would be non-local.

But there is a caveat. If we use graph partitioning naively, we will not be exploiting the fact that multiple CNOTs may be implemented using a single qubit. In what follows, we will explain how to make use of hypergraph partitioning, instead of simple graph partitioning, to account for this aspect. A more detailed review of hypergraph partition is given in Appendix~\ref{chap:HypPart}, here we summarise the key concepts:

\begin{itemize}
  \item Hypergraphs are the result of extending the definition of graphs to accommodate edges that may have a number of incident vertices other than two. More formally, a hypergraph is a pair of sets \((V,H)\), where \(V\) is the set of vertices and \(H \subseteq 2^V\) is the set of hyperedges. Each hyperedge is represented as the subset of vertices from \(V\) it connects. We will not consider any notion of directionality, i.e.\ all vertices of a hyperedge play the same role.
  \item Hypergraph partitioning follows the same premise as graph partitioning. The user provides the two parameters \((k,\varepsilon)\), which have the exact same meaning as before, and a hypergraph. What the problem now attempts to minimise is a metric known as \(\lambda\!-\!1\), which is defined as follows: given a partition of the hypergraph, the function \(\lambda\colon H \to \mathbb{N}\) pairs each hyperedge with the number of different blocks its vertices are in. Then, \(\lambda\!-\!1 = \sum_{h \in H} \lambda(h) - 1\) provides a measure of not only how many hyperedges are `cut' but also across how many blocks\footnote{Simply minimising the number of hyperedges `cut' is also an often used approach, but it is not as useful for our problem.}.
\end{itemize}

\subsection{Vanilla algorithm}

The one where only hyperedges from controlled qubits.

\subsection{Gate pushed extension}

The one where gates are pushed.

\subsection{Both ends are useful extension}

The one where we can also make hyperedges from target qubits.

\chapter{Implementation details and Results}
\chapter{Conclusions}

\section{Further work}


Better implementation (integrated in Quipper from a data-structure point of view; this should make the program faster)

Introduce more functionality:
  - CNOT interchange optimisation.
  - Black boxes: Sometimes we know some part of the circuit requires a lot of interaction between some qubits. If we tell the algorithm in advance to consider that circuit fragment as a black box which can not be distributed, the hypergraph should be way simpler, and therefore the partitioner should be able to find a better overall distribution.
  - Non-uniform partition: some QPUs get more vertices (this is trivial)

%%%%%%%%
%% Any appendices should go here. The appendix files should look just like the
%% chapter files.
\appendix
\chapter{Hypergraph Partitioning}
\label{chap:HypPart}

%% Choose your favourite bibliography style here.
\bibliographystyle{apalike}

%% If you want the bibliography single-spaced (which is allowed), uncomment
%% the next line.
% \singlespace

%% Specify the bibliography file. Default is thesis.bib.
\bibliography{Bibliography}

%% ... that's all, folks!
\end{document}
