\begin{figure}
\centering
\begin{tikzpicture}
  \pgfplotsset{ybar stacked}
  \begin{axis}[
      title={Varying \(k\in\{3,5,7,10\}\)},
      ylabel=\# / total CNOTs,
      height=200pt,
      width=350pt,
      bar width=7pt,      
      xtick={1,2,3,4,5},
      xticklabels={bf,bwt,gse,usv,qft200},
      ymin=0,
      legend pos=north west,
      legend style={font=\small},
      legend cell align={left}]
    \addplot [preaction={fill, black!30}, pattern=north west lines, pattern color=black] table [x=X, y=EbitsProp]  {Code/tests/data3.dat};
    \addplot [fill=black!30,] table [x=X, y=NonLocalPropDiff]  {Code/tests/data3.dat};
    \addlegendentry{Ebits};
    \addlegendentry{Non-local};
  \end{axis}
\end{tikzpicture}
\caption{For each circuit, the bars correspond from left to right to a value of \(k\): \(3\), \(5\), \(7\) and \(10\). Each bar shows the number of non-local CNOTs and ebits required, normalised over the total number of CNOTs the original circuit had. The results shown are for the best combination of enabled extensions in each case.}
\label{fig:results3}
\end{figure}
