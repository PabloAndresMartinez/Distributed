Quantum computers are more powerful than the computers we use nowadays. This difference is reflected in the fact that quantum computers could be used to make calculations that our current computers can not possibly solve in a reasonable amount of time. Small quantum computers have already been built, but for a quantum computer to be useful, it should be at least ten times larger than the ones we can currently build. However, building large quantum computers is extremely challenging, and there is still a long way until we develop the technology that would be needed.

There is an alternative to building a single large quantum computer, and it is to create a spider web of smaller ones that cooperate to perform calculations. This approach is commonly known as \textit{distributed computing}, and in this thesis we aim to make it more accessible to quantum computer scientists. We propose an automated translator that adapts calculations meant to be solved by a single quantum computer so it can be performed by a group of smaller quantum computers. In order to give a better intuition of the problem at hand, let's temporarily deviate from quantum matters and describe a situation we might encounter in real life, which is fairly similar to our problem: 

Imagine you are the human resources manager of a company that has just made a contract with a client. The contracted project is so demanding that all the employees of the company must work on it. The company has a large number of employees and you have to assign each of them to one of the multiple office buildings. The employees must work in groups, which are very interconnected, as each employee may work in multiple groups -- in mathematics, we call a network of interconnections such as this a \textit{hypergraph}. Now, the problem is that whenever you allocate to different buildings employees that work in the same group, you will need to hire an additional employee that communicates messages between the buildings. Clearly, you want to minimise the number of messengers you need to hire, but doing so requires a bit of planning. In fact, this is a difficult problem, known by computer scientists as the \textit{hypergraph partitioning} problem.

We can draw an analogy between the human resource problem and the problem of efficiently distributing a quantum program: Employees are \textit{qubits} -- the basic working unit of a quantum computer --, each building is one of the small quantum computers, and the messengers are a special kind of qubits. The company is the spider web of quantum computers itself, the client is the user, and the project contracted is the calculation the user wishes our computers to solve. What we propose in this thesis is an automated method for the resource manager to figure out what is the best way of allocating each qubit to the different computers so the least amount of messengers are required.

To do so, we make use of an already existent hypergraph partitioning solver. The subtlety here is that we do not know beforehand how the qubits have to work together to perform a given calculation; we do not know the hypergraph. The main contribution of this thesis is the automated generation of such hypergraph out of the description of the calculation given by the user. We give formal proof that an optimal partition of the hypergraph gives an optimal solution of how to distribute the qubits. Additionally, we propose two extensions that improve how good such a solution can be. %In our analogy, the first extension would correspond to allowing the human resources manager to plan the order in which subtasks of the project would be carried out. The second extension would loosely correspond to asking the human resources manager to decide in which building each messenger's office should be.

We used our method to distribute some programs scientists would be interested to execute on a quantum computer. The results were quite satisfying: Our method manages to make an efficient use of messengers, assigning each of them to multiple communications.