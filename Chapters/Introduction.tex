\chapter{Introduction}
\label{chap:Introduction}

In this thesis, we propose an algorithm for the automated translation of programs from monolithic quantum computers to distributed grids of quantum computers. The reader is not required to have any prior knowledge of quantum computing, as we will introduce the concepts needed in this thesis in an incremental manner. In some occasions, we will use analogies from classical computer science -- ranging from distributed architectures to compilers -- that should help understand the key concepts of distributed quantum computing. Our main contribution consists in reducing the problem of efficiently distributing a quantum circuit to the problem of hypergraph partitioning, which is a problem thoroughly studied in the literature of classical computer science.

Quantum computing aims to take advantage of quantum mechanics to speed-up computations. There are many examples of problems that, although solvable by a classical computer, the time it would take to compute them is unreasonable in practice, regardless how large or fast your classical computer is. Some of these \textit{intractable problems} can be solved efficiently on a quantum computer. Well-known examples of such problems are:

\begin{itemize} 
\item \textit{Factorisation of large numbers}: In classical computers, all known factorisation algorithms take exponential time with respect to the input size. It is strongly believed that there is no way a classical computer can solve the problem efficiently -- in fact, we are so confident about it that most widely used encryption systems, like RSA, rely on this. However, quantum computers are capable of solving that problem in polynomial time (i.e.\ making it tractable), using Shor's algorithm~\citep{Shor}.
\item \textit{Unstructured search}: The aim is to perform a brute-force search (i.e.\ requiring no prior knowledge about the search space) over \(N\) data-points. Classical computers have no other option than testing each data-point, so the time they take to perform the search is proportional to \(N\). With a quantum computer, using Grover's algorithm~\citep{Grover}, the search is done in time proportional to \(\sqrt{N}\).
\end{itemize}

Besides, a recent result by \citet{BQPSepPH} gives formal proof of the existence of a large family of problems that a classical computer may never solve in polynomial time, but are solvable in polynomial time on a quantum computer. Nevertheless, all of these results are theoretical in nature, and giving experimental evidence of this gap in computational power is a highly active area of research, known as \textit{quantum supremacy}. 

Quantum computing would be very valuable in many areas of research that deal with problems that are intractable on classical computers. Some of the main applications that have been discussed in the literature are:

\begin{itemize}
\item \textit{Chemistry, medicine and material sciences}: Calculating molecular properties on complex systems is extremely demanding for classical computers. However, polynomial algorithms for this kind of problems are known for quantum computers~\citep{TowardsQuantumChemistry}. %Quantum computers are likely to trigger a revolution on areas of science that need to model molecules and their interactions.
\item \textit{Machine learning}: Finding patterns in a large pool of data is the essence of machine learning. Multiple quantum algorithms have been shown to be able to detect patterns that are believed not to be efficiently attainable classically~\citep{QuantumMachineLearning}.
\item \textit{Engineering}: Optimisation and search problems are common in almost every area of engineering. Quantum computers are particularly well suited for these tasks, with Grover's algorithm (unstructured search) being an obvious example.
\end{itemize}

For any of these applications we will require large scale quantum computers. Due to the obstacles in the way of building a large quantum computer, some experts have advocated the alternative of building a \textit{quantum multicomputer}: a grid of small quantum computers that cooperate to perform an overall computation~\citep{DistributedQCHW}. Small quantum computers have already been built, and the technology needed to connect them in a grid is known. Thus, it is possible that large scale grids of quantum computers become available before large monolithic quantum computers do. Even if this were not the case, distributed quantum computing would still be a desirable asset, for similar reasons it is for classical computing: a cluster of small computers may be cheaper than a large mainframe, and a multicomputer is intrinsically modular, so we may easily adjust the amount of resources we dedicate to a computation.

Unfortunately, there is virtually no support for developing programs to be run on distributed quantum architectures. The literature on quantum algorithms and quantum programming languages focuses almost exclusively on a single model of quantum computation -- the circuit model --, meant to be run on a monolithic computer. In this thesis, we intend to make a first step towards the spread of distributed quantum algorithms, providing an automated method for distributing any input circuit across an arbitrary number of quantum computers.

\section{Outline of the chapters}

Chapters~\ref{chap:Overview} and~\ref{chap:Distributed} correspond to the literature review relevant to this thesis, and they identify the problem we intend to solve. Chapter~\ref{chap:Overview} introduces the key concepts of quantum computer science that will be required throughout this thesis. Most importantly, \S\ref{Principles} explains the principles of quantum computing, and gives an intuition behind the speed-up it achieves; while \S\ref{Hardware} discusses the scalability challenges of quantum computers, and presents the main model of quantum computing -- the circuit model. Chapter~\ref{chap:Distributed} gives an introduction to distributed quantum computing: In \S\ref{Ebits} we explain how quantum information may be communicated across QPUs, \S\ref{IntroDistributing} reviews the main paper our thesis is built upon~\citep{NonLocalCNOT}, and \S\ref{DQC_Architecture} describes an abstract distributed quantum architecture.

Our contributions are presented in Chapter~\ref{chap:Project}. First, in \S\ref{NonLocalGates} we propose two extensions of the work by \citet{NonLocalCNOT}. \S\ref{EfficientDistrib} is the core of the thesis, where we describe our algorithm for automated distribution of quantum circuits. We present our first version of the algorithm, then we describe its two extensions, corresponding to the ones we proposed in \S\ref{NonLocalGates}. \S\ref{Interchange} outlines a third potential extension that we propose as further work.

Finally, in Chapter~\ref{chap:Results}, we present the results of using our algorithm to distribute different benchmark circuits drawn from the literature. We discuss these results and study the effect of both extensions of our algorithm. Chapter~\ref{chap:Conclusions} ends the thesis by drawing some conclusions and suggesting further work.