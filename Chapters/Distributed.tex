\chapter{Distributed Quantum Computing}
\label{chap:Distributed}

As we discussed in \S\ref{Hardware}, there are different approaches on how to build quantum computers. Now that many of these have been experimentally demonstrated, having built small quantum computers, the question of how to scale up is increasingly relevant. This has led to the proposal of distributed architectures~\citep{ArchitectureSurvey}.

In classical computing, an standard example of a distributed computer is the Non-Uniform Memory Access (NUMA) architecture: A system of independent computing nodes, each having its own local memory. In order to collaborate to perform an overall computation, the different nodes will need to communicate. In NUMA, they do so by accessing each other's memory. While nodes can manage their own local memory efficiently, accessing another node's memory is slow. Hence, we always attempt to minimise the amount of communication between nodes. A distributed quantum computer would follow the same principles, where each quantum processing unit (QPU) would own a collection of qubits (its local memory) and may access another QPU's qubits at the cost of some overhead, using entanglement.

\section{Communication through entanglement}
\label{Ebits}

For a QPU to be able to access another's qubit, we must provide them with some sort of communication channel. Simply using a classical channel (sending bits) is not helpful: the whole point of representing the state of the computation in qubits is that they may be in a superposition of classical states, which would take up to an exponential amount of space and processing on classical bits. We could consider physically transporting the system that encodes the qubit from one QPU to another, and while that is certainly possible with photons, in general it is not feasible to have a channel that is both fast and protects well against information loss due to decoherence.

In \S\ref{Principles}, we explained it was possible to affect a distant qubit by acting on another qubit with which it was entangled. We wish to exploit this property in order to allow a QPU to query another's QPU qubit. There are different levels of how strong a pair of qubits is entangled -- intuitively, how much they affect each other. This is often formalised as the correlation between the qubits possible measurement outcomes; for instance, the pair of qubits \(\frac{1}{\sqrt{2}}\ket{0,0} + \frac{1}{\sqrt{2}}\ket{1,1}\) is said to be \textit{maximally entangled}, as the possible measurement outcomes are exclusively either \(\ket{0,0}\) or \(\ket{1,1}\), always matching in both qubits\footnote{\, In total, there are four \textit{maximally entangled} states of a pair of qubits: \(\frac{1}{\sqrt{2}}\ket{0,0} + \frac{1}{\sqrt{2}}\ket{1,1}\) and \(\frac{1}{\sqrt{2}}\ket{0,0} - \frac{1}{\sqrt{2}}\ket{1,1}\) give perfect correlation, while \(\frac{1}{\sqrt{2}}\ket{0,1} + \frac{1}{\sqrt{2}}\ket{1,0}\) and \(\frac{1}{\sqrt{2}}\ket{0,1} - \frac{1}{\sqrt{2}}\ket{1,0}\) give perfect anti-correlation}. Naturally, the most efficient communication channel will take advantage of entanglement in its strongest form, and so we will make use pairs of qubits entangled in this particular maximally entangled state. This qubit pair configuration is generally known as a Bell state, and Figure~\ref{fig:bell} shows how to prepare it.

\textbf{TODO}: fig:bell, show the circuit to make a bell state, how it is represented usually as a wire C and its Diract state.

An interesting property of the Bell state is shown in Figure~\ref{fig:sliding}: if a quantum gate, whose matrix representation is symmetric, is applied to one of the qubits, it is the same as if the gate was applied to the other qubit. In some sense, the gate can `slide' through the entanglement, like beads on a string; as if the entangled state were a curved wire, connecting the pair of qubits. Hopefully, this serves as a first intuition of how Bell states are a natural choice for implementing quantum comunication.

\textbf{TODO}: fig:sliding, show how a H gate can slide through an ebit.

So far, we have explained how to generate a Bell state inside a QPU (as in Figure~\ref{fig:bell}). However, what we aim for is that two different QPUs each own one of the qubits from the Bell state. The challenge is then to send one of the qubits to another QPU, while preserving the entangled state. The problem of sharing a Bell state between two parties is solved by the \textit{entanglement distillation} protocol~\citep{DistillationProtocol}, which ensures the Bell pair is transmitted to their destination with an arbitrarily small error factor. We will now give a brief description of this protocol.

\subsection{Entanglement distillation}
\label{Distillation}

\textbf{TODO}: This section

Sending an arbitrary quantum state across a channel is difficult, we could do it either fast or reliably, but not both. But if you do know what you are sending, you just send a lot of them.

Non-fidelity. If the ebit is not exactly in a state of the form \(\alpha\ket{0,0} + \beta\ket{1,1}\) (i.e.\ it has a non-zero coefficient for \(\ket{0,1}\) or \(\ket{1,0}\))

Often in distributed quantum computing literature, a Bell state shared by two QPUs is known as an \textit{ebit} (entangled-bit). We will also use this convention, and refer to each of the two qubits in the Bell state as the two \textit{halves} of the ebit.

\section{Distributing circuits}
\label{IntroDistributing}

We now explain how ebits are used to allow a QPU to peek into another's qubits. Here, we will introduce the proposal of \citet{NonLocalCNOT}. Later on, in \S\ref{NonLocalGates}, we will extend this work with our own contributions. 

We aim to split a given circuit and distribute the fragments across multiple QPUs. The gates that should operate over qubits on different QPUs are known as \textit{non-local} gates. As we previously mentioned, any circuit can be converted to Clifford+T circuit (thanks to the Solovay-Kitaev theorem). In Clifford+T, the only gate that operates on more than one qubit is the CNOT. Hence, we only need to understand how CNOTs can be implemented non-locally.

\textbf{TODO:} A figure with a simple circuit and dashed lines identifying how we want to split it, and highlighting a non-local CNOT.

The construction we will use is a slight variation of what was proposed by \citet{NonLocalCNOT}, and it is shown in Figures~\textbf{TODO}. In principle, we will use an \textit{ebit} per non-local CNOT. We will call the QPU that holds the target qubit (the one with a \(\oplus\)) the `target QPU' and similarly for the control qubit. 

\textbf{TODO}: showing the overall scheme, with cat-entangler and cat-disentangler as black boxes.

The implementation of a local CNOT gate has three steps. First, we must apply what the authors refer to as the \textit{cat-entangler} (Figure~\ref{fig:cat-entangler}), which creates a local `copy'\footnote{\, Note that there is no such thing as `copying' a quantum state (due to the non-cloning theorem). What we mean here by `copying' is generating, from \(\ket{\psi} = \alpha\ket{0} + \beta\ket{1}\), the state \(\ket{\psi} = \alpha\ket{0,0} + \beta\ket{1,1}\) which is fundamentally different from an actual copy: \(\ket{\psi,\psi} = \alpha^2\ket{0,0} + \alpha\beta\ket{0,1} + \alpha\beta\ket{1,0} + \beta^2\ket{1,1}\).} of the control qubit inside the target QPU. In the process, the ebit half in the control QPU is measured (and thus destroyed), and the outcome is used to correct the other half, in the same spirit as in the MBQC model. Notice that the only information physically crossing the boundary between blocks is the \textit{classical} outcome of the measurement (a bit, either \(0\) or \(1\)).

\textbf{TODO}: Show the circuit for the cat-entangler and the resulting state

Then, the CNOT gate is \textit{applied locally} inside the target QPU, between its ebit half and the target qubit. At the end, the \textit{cat-disentangler} must be applied (Figure~\ref{fig:cat-disentangler}), which simply destroys -- with a measurement -- the remaining ebit half and then corrects the control qubit, so the randomness of the measurement is counteracted. Once again, only classical information crosses the boundary.

\textbf{TODO}: Show the circuit for the cat-disentangler and the resulting state

In this way, we have implemented a non-local CNOT gate using one ebit and two classical bit messages between QPUs. However, the true advantage of this approach is attained when multiple non-local CNOTs are implementing using a single ebit: Once the cat-entangler is applied, any number of CNOTs whose target is in the same QPU, and that are controlled by the same qubit, may all be implemented by using the same ebit half as control, as shown in Figure~\ref{fig:multipleCNOTs}.

\textbf{TODO}: Show the circuit for many CNOTs, now with cat-entangler and everything

Now, depending of how we choose to partition the circuit, there will be different groups of CNOTs that we may be able to implement using a single ebit. We will then wish to find the partition that requires the fewest ebits to implement all of its CNOT gates. This optimization problem is not discussed in the original paper, nor in any other work, as far as we know. It will be our main contribution in this thesis, along with an extension of the results just explained, both found in Chapter~\ref{chap:Project}.


\section{Distributed quantum architectures}
\label{DQC_Architecture} 

A distributed quantum computer will have multiple quantum processing units (QPUs), each managing a small collection of qubits as their local memory. It should also have a subsystem specialised in generating and sharing ebits. As we discussed in \S\ref{Distillation}, there is a compromise between the quality of the ebit -- the quality of the communication channel -- and the time it takes to prepare it. Fortunately, \citet{NoisyChannels} showed that efficient distributed quantum computation using noisy ebits is feasible. Still, the generation of \textit{ebits} remains the main bottleneck of the architecture, and thus we will want to minimise the required ebit count as much as possible. 

\textbf{TODO} Detail the abstract distributed quantum architecture.

\textbf{TODO} Authors such as \citet{DistributedQCHW} have discussed the experimental construction of a similar architecture. It is a multicomputer that uses photons for communication across QPUs, entangling stuff. 